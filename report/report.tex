\documentclass{article}
\usepackage[utf8]{inputenc}
\usepackage{authblk}
\usepackage[colorlinks]{hyperref}
\usepackage[
	backend=biber,
	style=apa
]{biblatex}
\usepackage{filecontents}

\title{%
	$n$-body Simulator \\
	\large{Astro 101 Creative Project Report}
}
\author[1]{Giancarlo Pernudi Segura}
\affil{pernudi@ualberta.ca}
\author[2]{Zeyu Li}
\affil{zeyu7@ualberta.ca}

\begin{filecontents}{report.bib}
@ARTICLE{chenciner_montgomery_2000,
       author = {{Chenciner}, Alain and {Montgomery}, Richard},
        title = "{A remarkable periodic solution of the three-body problem in the case of equal masses}",
      journal = {arXiv Mathematics e-prints},
     keywords = {Mathematics - Dynamical Systems},
         year = 2000,
        month = oct,
          eid = {math/0011268},
        pages = {math/0011268},
archivePrefix = {arXiv},
       eprint = {math/0011268},
 primaryClass = {math.DS},
       adsurl = {https://ui.adsabs.harvard.edu/abs/2000math.....11268C},
      adsnote = {Provided by the SAO/NASA Astrophysics Data System}
}
\end{filecontents}

\addbibresource{report.bib}

\begin{document}

\maketitle

The final project can be viewed at \url{https://zeyu-li.github.io/black-hole/}. The code is fully open source under the \href{https://raw.githubusercontent.com/Zeyu-Li/black-hole/main/LICENSE}{GPL-3.0 license} and can be found \href{https://github.com/Zeyu-Li/black-hole}{here}. The website has only been tested on chromium based browsers such as chrome and brave, but also on firefox. It has NOT been tested on safari or mobile browsers. We highly suggest you use chrome or firefox for the best experience.

\section{Project Remarks}
Creating an n-body simulation is very complicated. In order to implement a simulator capable of running on a web browser, a couple of simplifications have been made. For one, once a body is reasonably outside of the bounds of the screen, it is deleted in order to save memory. Whenever two objects get really close, they merge immediately while mostly conserving momentum, like if two black holes collided such that there is no big explosion. Sometimes they may still not fully collide and instead catapult each other away.

\section{Simulation Limitations}
There are also limitations to accuracy that we cannot control. The law of conservation of energy cannot be maintained due to small rounding errors that come with computer floating point numbers. You may notice this as momentum may not be fully conserved when two bodies collide. We also didn't implement any gravitational radiation since the formulas are beyond the scope of this course.

\section{Use}
\subsection{Presets}
\subsubsection{Preset 1}
The first preset is two bodies of with the same mass orbiting around a barycenter.

\subsubsection{Preset 2}
The second preset is two bodies with different mass orbiting around a barycenter, with the first mass being 4 times bigger than the second mass.

\subsubsection{Preset 3}
The first preset is Three bodies in an 8-figure based on \cite{chenciner_montgomery_2000}. It will shift off-screen over time.

\subsubsection{Preset 4}
The first preset is Anohter variation of three bodies in an 8-figure based on \cite{chenciner_montgomery_2000}. It will shift off-screen over time.

\subsection{Random Body Generation}
You can also generate a body at a random location on screen by setting the mass, initial speed, and initial angle (for speed).

\printbibliography

\end{document}
