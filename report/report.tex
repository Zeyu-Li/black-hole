\documentclass{article}
\usepackage[utf8]{inputenc}
\usepackage{authblk}
\usepackage[colorlinks]{hyperref}

\title{%
	$n$-body Simulator \\
	\large{Astro 101 Creative Project Report}
}
\author[1]{Giancarlo Pernudi Segura}
\affil{pernudi@ualberta.ca}
\author[2]{Zeyu Li}
\affil{zeyu7@ualberta.ca}

\begin{document}

\maketitle

The final project can be viewed at \url{https://zeyu-li.github.io/black-hole/}. The code is fully open source under the \href{https://raw.githubusercontent.com/Zeyu-Li/black-hole/main/LICENSE}{GPL-3.0 license} and can be found \href{https://github.com/Zeyu-Li/black-hole}{here}.

\section{Project Remarks}
Creating an n-body simulation is very complicated. In order to implement a simulator capable of running on a web browser, a couple of simplifications have been made. For one, once a body is reasonably outside of the bounds of the screen, it is deleted in order to save memory. Whenever two objects get really close, they merge immediately while mostly conserving momentum, like if two black holes collided such that there is no big explosion. Sometimes they may still not fully collide and instead catapult each other away.

\section{Simulation Limitations}
There are also limitations to accuracy that we cannot control. The law of conservation of energy cannot be maintained due to small rounding errors that come with computer floating point numbers. You may notice this as momentum may not be fully conserved when two bodies collide. We also didn't implement any gravitational radiation since the formulas are beyond the scope of this course.
\end{document}
